􏰁 Que mettez-vous dans le champs Timestamp, et quelle utilisation en faites-vous ?
􏰁 Comment réagissez-vous à la réception de paquets PTYPE_NACK ?
􏰁 Comment avez-vous choisi la valeur du retransmission timeout ?
􏰁 Quelle est la partie critique de votre implémentation, impactant la vitesse de transfert?
􏰁 Quelles sont les performances de votre protocole? (évaluation graphique et/ou chi􏰂rée, scénarios explorés, explication de la méthodologie)
􏰁 Quelle(s) stratégie(s) de tests avez-vous utilisée(s) ?

\subsection{Introduction}
Ce document est un rapport détaillant certains aspects de notre projet pour le cours de réseaux informatiques. Il tend
à préciser certains choix de conceptions et montre en quelques lignes le déroulement du programme. Pour toutes informations
concernant l'utilisation du programme et son lancement en ligne de commande, rendez-vous dans le fichier Readme.

\subsection{Structure générale}
Vous trouverez dans ce dossier /src deux fichiers avec l'extension '.c'. Il s'agit de sender.c et receiver.c qui sont les deux fonctions
à appeler pour utiliser le chat. Vous trouverez dans ce même répertoire un autre dossier, /tools, contenant les différentes
fonctions relatives aux sockets, aux paquets et aux listes chaînées que nous avons implémenté.

\subsection{Déroulement du programme}

\subsubsection{Sender.c}
Le fichier sender.c gère l'envoie de données vers le receveur. Nous utilisons le sliding window protocol et le selective
repeat. La fenêtre est représentée par une liste simplement chaînée et sa taille est fixée à 32. Pour transmettre les
datagrammes, nous utilisons une socket.
Le sender procède en trois étapes. Dans un premier temps, il cherche à lire l'information et procède de la manière suivante:
\begin{itemize}
  \item Si il y a de la place dans la liste:
    \begin{itemize}
      \item recupération d'un morceau d'au plus 512 bytes (le payload).
      \item création d'un paquet et encodage du header. Le type du paquet est initialisé à P_TYPE_DATA et l'attribut
      \textit{length} est fixé à la taille du morceau récupéré. On fixe un numéro de séquence dans l'intervalle [0,255] -
      le numero de séquence est une variable de notre programme et est incrémentée à chaque attribution (on utilise modulo
      pour rester dans la bonne gamme).
      \item encodage du paquet (header + payload) dans une chaîne de caractère et écriture sur la socket.

Dans un deuxième temps, le sender va lire sur la socket en quête de feedback du receveur. En supposant que de l'information
y soit effectivement écrite, voici son fonctionnement:

\begin{itemize}
  \item Décodage de la chaîne de caractère afin de constituer un paquet.
  \item Evaluation du type du paquet (ACK ou NACK)

    \textbf{ACK :} Si c'est un ack, cela signifie que le feedback est positif. Deux cas de figures peuvent toutefois se
    produire. Imaginons que nous ayons reçu un ack avec comme numéro de séquence 3. Cela signifie que le receveur attend
    le paquet contenant le numéro de séquence 3. Mais a-t-il
    //HMMMM Plus trop sur de cette section

    \textbf{NACK:} A la réception d'un NACK, le sender va chercher le paquet en question dans la liste qu'il entretient.
      Lorsqu'il le trouve, il met l'attribut timestamp à jour et renvoie le paquet. Dans le même temps, il supprime chaque
      paquet contenant un numéro de sequence inférieur à celui du paquet non-acquis, ceux-ci ont bien été récupéré par le
      receiver.

// J'ai une question :
// Pour gérer la sliding window, dans mon esprit au debut c'était que la liste commencait a un numéro de sequence,
// tant qu'on a pas recu de ack avc ce numero de seq on retire aucun paquet apres et donc on est sur que la fenetre
// sera tjrs au bon endroit vu qu'on retire pas le premier.
// Par contre, je crois que jean voulait faire differemment, en mémorisant le seqnum du premier paquet, en additionnat avec
// la taille de la fenetre et en sachant quelle seqnum max pouvait entrer dans la fenetre (condition a verifier au moment
// de lire des datas dans le sender alors). L'un ou l'autre me vont mais different lors de la reception d'un ack. a discuter

Enfin, la troisième et dernière étape du sender est de lancer une routine afin d'évaluer quels paquets ont expiré.
Cette routine remplie sa mission en comparant la valeur de l'attribut timestamp avec "l'heure" actuelle. Si la différence
est supérieure à la valeur du retransmission timeout alors le paquet doit être ré-envoyé (avec une mise à jour de son
timestamp). La routine boucle sur la liste tant que le paquet rencontré a expiré. Puisque les paquets sont ajoutés
chronologiquement, les paquets sont triés du plus ancien au plus récent.






\subsubsection{receiver.c}
 La fonction receiver est assez simple. Le receiver entretient également une liste de paquet ainsi qu'une variable contenant
 le paquet attendu.

 Elle cherche à lire de l'information sur une socket. Dès qu'un client est trouvé,
 le receveur est connecté à son sender. Ensuite, on procède comme suit:
 \begin{itemize}
 \item Décodage de la chaîne de caractère afin de créer un paquet.
 \item Si l'attribut tr du paquet est à 1, cela signifie que le paquet est tronqué. On ne peut pas envoyé un nack pour ce
 paquet. En effet, puisqu'il est tronqué, les données qu'il contient ne sont pas fiables et par conséquent le nack serait
 une potentielle source d'erreur.
 \item Si le paquet n'est pas tronqué, le receiver regarde si le numéro de séquence du paquet est celui attendu.

 Si tel est le cas :
 \begin{itemize}
  \item Le receiver récupère le payload et le dirige sur la sortie standard puis, si l'option -f est mentionnée, redirige
  le flux vers le fichier spécifié.
  \item envoi d'un ack contenant le timestamp et le numéro de séquence de la donnée reçue.
  \item incrémentation de la variable indiquant le paquet attendu
  \item parcours du buffer pour identifier si le paquet attendu si trouve. Si tel est le cas, les trois opérations ci-dessus
  sont d'application et la place du paquet est libérée.

Si ce n'est pas le cas :
\begin{itemize}
  \item Le paquet est stocker en queue du buffer.
  \item Un Nack contenant le nuémro de séquence de la donnée attendue et le timestamp du paquet reçu.


\subsection{Choix de programmation}
\paragraph{Quelle valeur avons-nous choisi pour le retransmission timeout?} //ICI ON DEVRAIT PROCEDER PAR ESSAI ERREUR
\paragraph{Que metttre dans le champs Timestamp, et quelle utilisation en faire ?}
\paragraph{Pourquoi une liste et pas simplement un buffer}


\subsection{Tests}
\subsection{Performance}
\subsection{Conclusion}
